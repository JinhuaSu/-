\documentclass{ctexart}

% Language setting
% Replace `english' with e.g. `spanish' to change the document language
% \usepackage{authblk}
\usepackage[english]{babel}

% Set page size and margins
% Replace `letterpaper' with`a4paper' for UK/EU standard size
\usepackage[letterpaper,top=2cm,bottom=2cm,left=3cm,right=3cm,marginparwidth=1.75cm]{geometry}
% Useful packages
\usepackage{amsmath}
\usepackage{graphicx}
\usepackage[colorlinks=true, allcolors=blue]{hyperref}

\title{《资本论》所预言的:主流科幻作品对未来的悲观描述}
\author{苏锦华  (统计学院) 2021103740}


\begin{document}
\maketitle

\begin{abstract}
      《资本论》对未来有预见性的判断,但也仍然有它的时代局限性。
      本读书笔记主要针对《资本论》第一卷“相对剩余价值产生”章节涉及的内容进行探讨,针对现代网络空间、虚拟社群、线上办公的发展,结合赛博朋克、元宇宙等科幻概念,尝试讨论其中的不变与变化。读书笔记的结构主要围绕着生活资料与电子产品消费、算力与非中心化劳工关系、网络与相对剩余价值的生产,虚拟世界与游戏劳动,高科技低生活的赛博朋克文化上述话题,话题由现代生活逐渐牵引向科幻中的未来,以科技新热点来讨论《资本论》中经典论述的前瞻性以及可能尚待进一步研究的空白之处。
\end{abstract}

\section{生活资料与电子产品消费}


对于资本家来说,支付给工人的工资需要满足工人购买基本的生活资料,否则难以完成生产力的再生产\cite{马克思2004资本论}。以手机为代表的电子消费产品,已经深入生活的方方面面,成为衣食住行、生产活动的必要组成部分,不少消费电子产品甚至承担了部分生产资料的职能,使用电子产品的熟练程度以及高级快捷技巧甚至影响着个人生产力的高低。

电子消费产品在生活资料以及生产资料的界限逐渐模糊,因为无法离身以及承担网络身份的接入的属性,甚至可以算是一种的“新器官”,无电既是衰竭,衰竭“牵动”人的心理稳定。以手机为例,研究者史为恒对500名的大学生的调查发现,有六成的的大学生承认自己对手机“比较依赖”。大学生对手机的依赖具体表现在以下几方面:即时通信应用使用成瘾,脾气越来越坏;宁愿选择手机聊天、不愿面对面沟通;手机成为寸步不离的“玩具”;在重度手机用户中,没有手机后,有相当一部分人出现了一些躯体化症状,身体出现莫名的痛苦;很多人的社会能力减弱,甚至出现了抑郁和焦虑等症状。这像是一种心理上的成瘾。和药物、烟酒成瘾类似,失去了手机也让人产生戒断反应。在智能手机集大成的功能一体化的趋势下,人们能够使用多任务功能运行多项程序,对于大部分自我控制意识淡薄的大学生群体来说,在生活学习压力较大的环境下,手机使用成瘾的概率将进一步提高。此外,手机成瘾所带来的隐形伤害是专注度的下降。目前,越来越来的手机应用更加注重用户界面的漂亮设计以吸引使用者的注意力和下载量,这往往降低了信息本身的重要性。而对于一些不需要懂很多脑筋去思考的应用,此类接触会使得大学生长时间不动脑筋和没有了足够的想象空间而止于过于直观的感受。若大学生使用者习惯于这些连续不断的官能刺激,将会对显示中的如课堂内容和相对抽象的理论学习带来阻碍和不变,缺乏耐性最终导致注意力缺乏。而手机所具备的多任务功能使得人们习惯于“三心二用”,长期不能集中注意力将大大削减了我们思考的能力,让我们的思维难以深入到更加复杂的层面。

我们知道,马克思的资本积累理论把相对剩余价值生产看作是资本主义经济具有特征性的剥削手段,认为资本家会不断变革生产方法,提高劳动生产率,以获取额外剩余价值和相对剩余价值\cite{高峰2004产品创新与资本积累}。由此必然推动资本的技术构成和有机构成的上升趋势,而资本有机构成的上升则成为一般利润率趋于下降的根本条件。
数字货币、电子支付等产品创新已经全面渗透资本家卖出产品获得收入,资本家支付劳工工资、劳工使用工资购买生活资料的方方面面,也就是说无论是必要价值还是剩余价值,都通过电子支付渠道进行流通。从效率的角度来说,一定程度上加速了整个社会的生产周期,减少了支付相关所花费的必要劳动时间,使得相对生产时间增加。
但任何重要的技术创新和新兴产业部门,在经过一个较长时期的发展之后,当其技术的应用得以普及,或产品的需求趋于饱和,其发展的潜力就会逐渐衰竭。若是有任何人反对这种趋势,不使用电子产品,都会被资本家或者社会以“帮助”的方式进行裹挟,比如对于不少老人来说,甚至是出于买菜只能使用电子设备的原因,花费时间和金钱购买智能手机、学习使用微信以及支付宝。这时投资对收入的负乘数效应和收入对投资的负加速效应都将起作用,生产与消费的矛盾也将重新突出起来。

从这个角度来说,由于电子设备以及颠覆性的电子技术的应用,生活于社会的固定成本增加,这个成本随着科技的发展会不断攀升。中央美术学院赵厚的动画短篇《面具国》讲述一个架空的世界,这里戴面具被写入法律,人们生活在面具的下面,摘掉面具就会被视为异类,并且会受到面具的追捕。主角试图反抗,最终被抓起来,扔回到原来的生活中。类似作品出现在赛博朋克类型作品较为多见,这些人工智能元素和虚拟现实元素的构建某种程度上也是对现实世界的映射\cite{栗桢2013关于赛博朋克电影主要元素的哲学思考}。生物科技的发展使得提升人各项能力的电子义肢成为生活生产的必要工具,这些工具使得人各项原始技能逐渐萎靡,只保留群体化的、易被资本家操作的基础能力--人性。现如今各大工厂或企业的“智能代工”现象越来越普遍,很多人担心这会导致大量劳动者丢掉维持生计的饭碗\cite{张爱丹2019人工智能时代就业问题的伦理思考}。尤其是随着人工智能技术的不断向前发展,即从弱人工智能到强人工智能,乃至将来可能创造出的超级人工智能,人类的整个就业的前景已然不被看好。虽然已经有不少人指出,人工智能在取代人类工作岗位的同时,也会给人们创造出一些新兴工作岗位。但是由于这些新的工作随后又会被更为先进的人工智能不断替代,因而随着人工智能时代的深入,必然存在人类就业岗位不断递减的大趋势。也就是说,被人们所担忧的所谓“社会失业潮”是一定会出现的。不过对于这个结果,我们根本无须有任何的悲观、恐惧和拒斥。因为人类就业岗位递减的大趋势,正是推进人的全面自由发展所必需的过程,它意味我们将拥有越来越多的自由时间,可以去做自己喜欢和感兴趣的事情,有更多的可能去施展自己的天赋和才华。同时,越来越强的人工智能,也会不断地和全面性地提高我们生活的质量,使我们在各个层次的需求的满足度也不断提升。更重要的是,人们可以通过对人工智能反思和建构人类自身,以及通过人工智能的共产主义应用,实现人机共生,实现人的自由个性发展\cite{庄忠正2019人工智能的人学反思}。

\section{算力与非中心化劳工关系}

列宁指出资本主义是垂死的,腐朽的。然后资本主义垂而不死,腐而不朽的思想在仍然流行。因为资本主义是思想层次的,并且在堆积算力能够出现奇迹的时代,人们驾驭算力、占有算力、出借算力的行为悄无声息地根植下资本主义的种子。因为算力以其泛用性,如同电力一般,成为重要的生产资源。在数字经济背景下的万物智联时代,传统的数据中心与终端的两级模式已然无法满足计算所需的要求,因此,必然要求算力从网络中心流向边缘进行扩散。因而,边缘处理的能力需求会在未来数年中急剧增长,构建由中心云、边缘智能和智能终端为主的三级处理体系架构,从而形成分布式算力基础架构。因此,有必要从底层架构上考虑和设计云、边、端深度融合方案,构建算力网络架构\cite{吕廷杰2021数字经济背景下的算力网络研究},这种架构下,人人都是算力的指挥官。

人人都是算力的指挥官时,而算力执行的事项越来越接近人类智能时,资本主义的思想就会潜移默化地影响着人对待其他劳工、机器的态度和思想。资本家之所以是资本家,并不是因为他是工业的领导人,恰恰相反,他之所以成为工业的司令官,只因为他是资本家。这里“只因为”显得过于天然性,思想的渐近性和发展性使得奴役算力的人更有可能认同资本主义的观点。

区块链技术以及加密货币的出现,为加密货币提供算力进行分布式记账将获得一定的代币报酬,使得非中心化的劳工关系出现。由于加密货币算法公开透明,人人算力参与的特点,不同于由政府背书掌控的法币,谁也无法垄断控制加密货币,但人人都可以通过提供算力为其余代币使用着提供记账的劳动。由于区块链是一种按照时间顺序将数据区块用类似链表的方式组成的数据结构,并以密码学方式保证不可篡改和不可伪造的分布式去中心化账本,能够安全存储简单的、有先后关系的、能在系统内进行验证的数据\cite{沈鑫2016区块链技术综述}。区块链的出现解决了数字货币的两大问题\cite{安庆文2017基于区块链的去中心化交易关键技术研究及应用}:双重支付问题以及拜占庭将军问题。双重支付问题是同一笔钱被使用了超过一次,是在原有的以物理实体(纸币)为基础的传统金融体系中自然可避免的问题。在区块链出现之前的数字货币,都是通过可信任的中心化第三方机构来保证,以前是银行,现在是支付宝、微信支付等。区块链技术通过共识机制和分布式账本,不需要可信第三方就可以解决双重支付的问题是数字货币的一大突破,形成去中心化的可信分布式系统,产生交易、验证交易、记录交易信息、进行同步等活动均是基于分布式网络完成的,是彻彻底底的去中心化。

社会的具体需求一方面使得专业分工化,而知识的扩散、代码开源运动以及对组织结构扁平化的推崇在要求人才的灵活性和全面性。非中心化的临时劳工关系变得极其常见,很多情况下还存在并行关系,很多仅在科研合作出现的合作关系也在不少互联网公司中出现。不同项目承担不同工作职位的普及型还在于算力的通用性以及社会分工下专业功能API接口按次收费的普及,使得过于专业的能力通过调用平台API即可达成初步解决。以智能合约为例,区块链技术的智能合约是一组情景——应对型的程序化规则和逻辑,是部署在区块链上的去中心化、可信息共享的程序代码。签署合约的各参与方就合约内容达成一致,以智能合约的形式部署在区块链上,即可不依赖任何中心机构自动化地代表各签署方执行合约。智能合约具有自治、去中心化等特点,一旦启动就会自动运行,不需要任何合约签署方的干预。这样的行为逻辑极大降低了沟通成本,也契合了分布式社会系统的概念,其中每个节点都将作为分布式系统中的一个自治的个体,随着区块链生态体系的逐步完善,自治节点通过更为复杂的智能合约参与各种去中心化应用,形成特定组织形式的去中心化自治组织和公司,最终形成去中心化自治社会。

\section{网络与相对剩余价值的生产}

剩余价值学说是马克思经济理论的基石, 是马克思划时代的伟大功绩。然而,当我们运用传统的剩余价值理论来考察分析现代企业相对剩余价值的生产时, 却遇到了难以解决的矛盾现象。传统剩余价值理论认为, 就相对剩余价值的生产来说, 生产工人的工作日一开始就分成必要劳动时间和剩余劳动时间这两个组成部分;为了延长剩余劳动时间, 就要用各种方法缩短生产工资的等价物的时间, 从而缩短必要劳动时间。这种在工作日长度已定的条件下由于缩短了必要劳动时间而相对地延长了剩余劳动时间所生产的剩余价值就是相对剩余价值。对企业而言, 只有变革劳动过程的科学技术条件和社会条件, 提高劳动生产率, 降低生活资料的价值, 从而降低劳动力的价值, 才能缩短必要劳动时间, 生产相对剩余价值。相对剩余价值的生产,是整个社会劳动生产率提高的结果, 进一步讲是社会上所有企业通过吸纳科学技术, 改进生产的科学技术条件, 提高劳动生产率的过程来实现的。简言之, 相对剩余价值是在社会各企业普遍采用科学技术来提高劳动生产率的前提下由“ 生产工人” 创造的。可以说, 我国经济学界传统理论对相对剩余价值的认识大致如此。相对剩余价值的生产过程, 实质上表现为伴随企业“ 整个科技创新过程” 的进行, 科学价值库中的潜在价值通过技术成果的吸纳和企业产品的凝聚等环节而最终在社会生产领域中加以实现的过程,或者说是科学价值库中的价值通过技术创新的成果并入企业生产并借助于企业中生产工人之“ 手”加以显化或转化出来的过程。这个过程,在现象层面表现为生产工人创造剩余价值的过程, 但在实质层面并非就是生产工人创造了这些相对剩余价值。既然相对剩余价值的生产过程就是企业整个科技创新的过程, 相对剩余价值的绝大部分不是由生产工人创造的, 而是伴随企业整个科技创新过程的进行而从科学价值库中的潜在价值转化出来的, 那么在现代大量的高新科技产业, 尤其是在类似于“无人工厂” 的企业中的高额利润, 表面上看是由“ 生产工人” 来创造的, 而实质上是从科学价值库中的潜在价值转化出来的\cite{刘冠军2006科技创新与相对剩余价值生产}。这样, 即便是在现代大量的高新科技产业或企业中真的达到了“ 无人” 的程度, 那仍然可以按照我们对相对剩余价值的新理解, 来解释、说明该种企业中的高额利润的来源问题。随着互联网等高新技术的发展,网络使得资本家有了提高劳动生产率和提高劳动强度的强力工具。机器的联网控制使得居家线上办公成为可能,线上会议也充分降低了沟通成本,在线协同编辑和版本控制工具使得文案、代码的迭代变得更加高效。

但网络也使得劳工监控变得前所未有的神经且荒诞,正如《1984》中“老大哥在注视你”这样的毛骨悚然的感觉。基督教徒们不一定得到自己真主的时刻关注,但他一定逃脱不了资本家的时刻关注,只要资本家想的话。无论是钉钉回复用时统计、上厕所时长报警装置、wifi监控员工流量使用情况还是微博上监控自家员工的不利言论,在网络技术发展的条件下员工在资本家面前毫无隐私可言。

网络同时也使得企业公关变得前所未有的重要,相比于企业经营风险,公关风险简直如暴风骤雨,足以让毫无准备的企业陷入泥沼。企查查等APP使得资本家的风险暴露给求职的普通劳工,伤害社会风俗的职场性丑闻的曝光也能使得资本家在网络上被舆论风暴痛击。民间自发的工会组织也有勇气积极伸张他们的权力,“996ICU”是一次现象级的事件,不少被压榨的互联网企业劳工借助互联网本身也能激发“自由引导人民”的壮举,甚至加速国家政策的落地实施。

网络作为媒体渠道使得个人的影响力被无限放大,足以以小博大,和资本家争夺定义“正义”的话筒。《V字仇杀队》讲述了在一个平行的时空里,纳粹赢了二次世界大战,英国沦为殖民地,一名自称“V”的蒙面怪客处处与政府为敌,借电视新闻媒体引发了被压迫人民的暴力反抗。同样近年《小丑》等漫画反派角色能造成如此大的影响也在于媒体,媒体作为魔鬼代言人,一方面可以影响企业形象、还可以从思想等各个层次影响个体。所以网购主播带动的经济不一定是过硬的推广产品,还有其刻入购买者潜意识某种虚假人设。如果把人设理解为IP,那么以自身为IP的大型影响力人物,其自身就是劳工与资本家的矛盾辨证体。本质上资本家可以完全不参与劳动将企业交由专业经理人经营,但是对于这种人设IP,自身不可能不参与劳动。

\section{虚拟世界与游戏劳动}

网络时代的中后期是精神文化服务,也就是虚拟服务经济。虽然虚拟世界距离技术成熟还遥遥无期,但如今元宇宙话题的兴起(最初源自于1992年的科幻小说《雪崩》,小说描绘了一个庞大的虚拟现实世界,在这里,人们用数字化身来控制)也值得我们注意。上一章节提及了魔鬼代言人以自身影响力打造人设IP是否能归属于资本家的问题,如果到达了网络中后期,这种情况可能将由另一种现代已经初见雏形的产物替代--虚拟主播。对于物质生活极大丰富的世界,精神的各种刺激可能比物质更具吸引力,毕竟对物质的消费需求是有极限的,正如恩格斯系数描述饮食消费一样。因为现实空间的有限性,购买空间以及占用实体空间的物体总是相对受限的,城市化趋势又是不可逆的。有限的工资购买游戏中的虚拟商品或者社交软件的虚拟会员变成了一种可能的消费热点。对于消费人设来说,虚拟主播有更大的优势,首先是虚拟人设永远不会死亡。因为虚拟主播背后的“中之人”是可以替换的,相比于银幕中的007演员的换人,虚拟主播“中之人”的替换在一定情况上甚至能让粉丝无法察觉,这样一来虚拟主播IP变成了一种机器,一种资本家的生产资料。

虚拟世界首先出现于游戏,游戏通常享有的市场不受地区保护主义限制,也就是说文化不通也能通过游戏进行一定程度的受限沟通。游戏也像电影、文学那样逐渐成为一种艺术形式,有极其商业化的骗氪游戏,但也有自带文化输出的珍宝游戏。不同于单机游戏,现代流行的具有社交属性的MO类游戏(Multiplayer Online Game)中并非所有玩家都是上帝版的客户,随着充值的不同,不充值的玩家虽然也可以进行游戏的游玩,但是在进行一种扩大化的“游戏劳动”(Playbour)。综观现代游戏的历史, 至少有两类行为曾被冠以游戏劳动的名号, 一是Modding(修改游戏);二是Gold Farming(打钱)。

游戏爱好者修改/增补游戏设计(Modding)并将改版出卖给游戏公司的行为可以视作“游戏劳动”。游戏商的目的不是寻求资本增殖的偶然性,比如忽然收买一个能让Modder发大财的游戏补丁,或者商品化某些DIY(Do It Yourself)达人的“特异能力”,而是试图建立一种新的常态化的剥削秩序,把游戏行为直接转换为可资剥削的活劳动,使所有游戏者都受雇为“愉快”的“游戏劳工”,永世为其产出剩余价值。比如大火的英雄联盟、王者荣耀的鼻祖游戏DOTA就曾是在暴雪游戏的魔兽争霸3下进行创意开发,也就是说玩家基于某款游戏提供的素材进行魔改得到的创意游戏其商业价值可能最终大于原游戏。MO类游戏在中国的发达,最初源于网络游戏的“免费模式”,认为“非人民币玩家”在相当程度上就是库克里奇所指的“游戏劳工”(Playbour),他们沉迷游戏的同时,也免费为游戏商吸引了用户,创造以及改善了游戏内容。

Gold Farming(打钱)被认为是游戏劳动的另一种疑似形态,它是比较常见、也比较极端的例子, 指一些游戏工作室专门雇佣廉价劳动力在MMO(Massively Multiplayer Online Game)类游戏里“打金币”,并用金币兑换现实货币的行为,这些行为的主体通常被称为Gold Farmer(金币农夫)。严格说来, Gold Farming不能算娱乐行为,而是阉割了娱乐性的以生产为直接目的雇佣劳动,MMO类游戏不过在降维世界为资本提供剥削活劳动的数字平台。如今成长速度最快的移动端游戏里,比如《王者荣耀》已取消了金币的硬通货地位,我们在这些放弃金融系统的游戏里看不到Gold Farming的存在,这是否意味着游戏劳动在新的游戏技术面前灰飞烟灭,MO类游戏的剥削结构因为游戏设计者的“良心”发现而不复存在呢?答案自然是否定的。我们之前看问题的方法存在问题,太过依赖经典马克思的论述范畴,而忽视了游戏劳动及其剥削的隐蔽性与弹性,需要审时度势,重新检讨游戏劳动的定义。

\section{高科技低生活的赛博朋克文化}

再进一步,未来有虚拟世界听起来是不赖的,那未来的现实世界是怎样的,科幻作品中的赛博朋克(Cyberpunk)文化给出了答案--高科技、低生活。《仿生人会梦见电子羊吗》可以算是这类作品的经典之作,电影《银翼杀手》也是据此改编,游戏《赛博朋克2077》也曾为这一小众的题材破圈。赛博朋克作品大多描绘在未来,建立于“低端生活与高等科技结合”的基础上,拥有先进科学技术,再以一定程度崩坏的社会结构做对比。拥有五花八门的视觉冲击效果,比如街头的霓虹灯、街排标志性广告以及高楼建筑等,通常搭配色彩是以黑、紫、绿、蓝、红为主。故事框架是以社会秩序受到政府或财团或秘密组织的高度控制,而主角利用其中的漏洞做出了某种突破。赛博朋克的情节通常围绕黑客、人工智能及大型企业之间的矛盾而展开,背景设在不远的将来的一个反乌托邦地球,而不是早期赛博朋克的外太空。它实际上标志着针对以往科幻小说不注重信息技术的具体设定的缺点的改善和进步。

赛博朋克的作品纷纷对未来给出了一种悲观的描述,大型企业意指垄断资本主义发展到一定地步,伴随着生物科技的发展以及人工智能的开发,机器人可以在很多工作上替代人,人可以装上各种机器义肢变得非人,实现不死。人的思想也可以上载(电影《Upload》也描述了一种悲观的未来虚拟技术发达的社会)到云端实现思想层面的不死,如《赛博朋克2077》中的强尼银手那样。资本家仍然存在,剥削的对象已经转变为人工智能和带义肢的物理强化人类。人类劳动赚钱为的就是购买更加高级的义肢,暴力、毒品、卖淫、犯罪仍然存在。甚至大型公司的资本家也可能不是人,是比人类大脑聪明得多的电脑系统。这样的社会,矛盾仍然是资本家和劳工的对立,但是生活资料以及生产资料已经混为一谈,智慧、力量、灵活在义肢改装下可以并存。劳工既可以是人工智能也可以是人类,资本家既可以是人工智能也可以是人。思想上的洗脑和控制是这种社会合理存在的唯一条件,cyber的英文愿意也是控制论、神经机械学。这种情况下,矛盾和对立既是可笑荒诞的又是绝望无解的。赛博朋克文化的意义在于敲响警钟,当然我们看到资本主义腐而不朽的同时,不能麻木不仁地仍由其将人类社会引导至赛博朋克的无解之域。



\bibliographystyle{alpha}
\bibliography{sample}

\end{document}